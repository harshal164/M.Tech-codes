\documentclass[12pt, a4]{article}
\usepackage[utf8]{inputenc}
\usepackage{listings}
\usepackage{graphicx}
\graphicspath{ {images/} }
\lstset{
	breaklines=true,
	columns=flexible,
	}
\begin{document}
\title{OpenSSL - Lab Assignment 2}
\author{Snehal Gharat}
\date{\today}
\maketitle

\section{OpenSSL Basics}
\begin{itemize}
\item OpenSSL provides a command line application to perform a wide variety of cryptography tasks, such as creating and handling certificates and related files.
\item We can obtain a list of available ciphers by using the \textbf{list-cipher-algorithms}  command  
\begin{lstlisting}
$ openssl list-cipher-algorithms
\end{lstlisting}
\item Options
\begin{itemize}
\item[1.] \textbf{-in filename}: Specifies the input file 
\item[2.] \textbf{-out filename}: Specifies the output file
\item[3.] \textbf{-e or -d}: Specifies whether to encrypt (-e) or to decrypt (-d).
\item[4.] \textbf{-a, -base64}: These flags tell OpenSSL to apply Base64 encoding before or after cryptographic operation
\item[5.] \textbf{-iv IV}: specifies the initialization vector IV as hexadecimal number
\item[6.] \textbf{-K key}: allows us to set the key used for encryption or decryption.
\item[7.] \textbf{-salt, -nosalt}: Allows to switch salting on or off.
\end{itemize}
\end{itemize}

\section{OBSERVATIONS}
\subsection{Electronic Code Book (ECB)}

\begin{itemize}
\begin{lstlisting}
$ openssl enc -des-ecb -in input.bin -out ecb_encrypt1.bin -a -salt -K 342f1982c2ee32
\end{lstlisting}
\item This command reads our message which is written in input.bin file, \textbf{encrypts it} and writes the output in ecbencrypt1.bin file.
\item Here we have used the key K as \textbf{342f1982c2ee32}
\item Using the \textbf{xxd input.bin} and \textbf{xxd ecbencrypt1.bin} we observe the hex dump codes of this files
% ecb image
\begin{center}
\includegraphics[scale=0.5]{ecb.png}
\end{center}

\item When we change some values in the input file, we can see the resulting change in corresponding output file. Here I have made changes in last digit of second line.
\begin{center}
\includegraphics[scale=0.5]{ecb2.png}
\end{center}
\item When we decrypt it we get the following results, we need to use the same key for decryption.
\begin{lstlisting}
$ opensecb -in ecb_encrypt1.bin -a -salt -K 342f1982c2ee32
\end{lstlisting}
\begin{center}
\includegraphics[scale=0.5]{ecb_d.png}
\end{center}
\end{itemize}

\subsection{Cipher Block Chaining (CBC)}
\begin{itemize}
\item Here we need to mention the intial vector (IV) option in the command.
\begin{lstlisting}
$ openssl enc -des-cbc -in input.bin -out cbc_encrypt1.bin -a -salt -iv 0000000000000000 -K 342f1982c2ee32
\end{lstlisting}
\item Here we observe that, only the first line of hex code is same as that of EBC mode hex code.
\begin{center}
\includegraphics[scale=0.5]{cbc.png}
\end{center}
\end{itemize}

\subsection{Cipher Feedback Mode (CFB)}
\begin{itemize}
\item As we can see that there are no similarities between CFB and ECB or CBC mode.
\item The encryption and decryption commands are -
\begin{lstlisting}
$ openssl enc -des-cfb -in input.bin -out cfb_encrypt1.bin -a -salt -iv 0000000000000000 -K 342f1982c2ee32

$ openssl enc -d -des-cfb -in cfb_encrypt2.bin -a -salt -iv 0000000000000000 -K 342f1982c2ee32
\end{lstlisting}
\begin{center}
\includegraphics[scale=0.5]{cfb.png}
\end{center}
\item Also observe the bits changed due to single error bit.
\begin{center}
\includegraphics[scale=0.5]{cfb_err.png}
\end{center}

\end{itemize}
\subsection{Output Feedback Mode (OFB)}
\begin{itemize}
\item Here we observe that, due to a single error bit only one bit is being changed in the encrypted file.
\begin{center}
\includegraphics[scale=0.5]{ofb.png}
\end{center}
\end{itemize}

\subsection{Counter Mode}
\begin{itemize}
\item This is not possible using \textbf{openssl enc -des} command
\end{itemize}

\end{document}